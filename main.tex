\documentclass[12pt]{article}
\usepackage[francais]{babel}
\usepackage[latin1]{inputenc}
\usepackage{xcolor}
\usepackage{bm}
\usepackage{fancyhdr}
\usepackage{bibunits}
\usepackage{gb4e}
\usepackage{enumitem}
\usepackage{tabto}

\pagestyle{empty}
\oddsidemargin 0.0in
\textwidth 6.5in
\topmargin -0.75in
\textheight 9.5in

\renewcommand{\labelenumi}{\arabic{enumi})}
\usepackage[T1]{fontenc} 
\begin{document}

\pagestyle{plain}
\fancyhf{}


%%%%%%%%%%%%%%%%%%%%%%%%%%%%%%%%%%%%%%%%%%%%%%%%%%%%%%%%%%%%%%%%%%%%%%%%%%%%%%%%%%%%%%%%%
\begin{center}
\vskip 1cm
{\bf \Large Curriculum Vitae: Cl\'ement Helsens} \\
\end{center}
\vskip 0.5cm
\hskip 14 cm January 2019
\vskip 1.8cm

%%%%%%%%%%%%%%%%%%%%%%%%%%%%%%%%%%%%%%%%%%%%%%%%%%%%%%%%%%%%%%%%%%%%%%%%%%%%%%%%%%%%%%%%%
{\bf \large Personal Data}
\vskip 0.3 cm
\TabPositions{2.5cm}
\begin{itemize}
\itemsep0em
%\item[] Name \tab : Helsens Cl\'ement
\item[] Birthdate  \tab : September 17, 1982
\item[] Birthplace \tab :   Roubaix (France) 
\item[] Citizenship \tab :  French
\item[] Civil status \tab : Married, 2 children
\item[] Address \tab : CERN, CH-1211 Gen\`eve 23, Suisse
\item[] Phone \tab : +41 22 767 1165  / +33 6 78 79 64 56
\item[] E-mail \tab : clement.helsens@cern.ch
\end{itemize}


%%%%%%%%%%%%%%%%%%%%%%%%%%%%%%%%%%%%%%%%%%%%%%%%%%%%%%%%%%%%%%%%%%%%%%%%%%%%%%%%%%%%%%%%%
\vskip 0.7 cm
{\bf \large Education and Research Positions}
\vskip 0.3 cm
\TabPositions{2.5cm}
\begin{itemize}
\itemsep0em
\item[] Aug 2015 - Jul 2020: Research LD Staff at CERN, Geneva, Switzerland.
\item[] Feb 2013 - Jul 2015: Research fellow at CERN, Geneva, Switzerland.
\item[] Nov 2009 - Jan 2013: Research associate at IFAE, Barcelona, Spain.
\item[] Oct 2006 - Oct 2009: Ph.D. student at CEA-Saclay, France.
%\item[] 2006 - 2009: Ph.D. thesis at the CEA-Saclay and University of Paris-Sud 11, Orsay. 
%\tab \tab Supervisors C. Guyot and H. Bachacou. \\
%\tab \tab "Search for high mass neutral gauge bosons decaying into a pair of muons\\
%\tab \tab in the ATLAS detector".
%\item[]June 2006: Master of Science at the University of Paris-Sud 11, Orsay%, \\
%\tab \tab Particle Physics
%\item[] 2003 - 2006: Bachelor in Physics at the University of Paris-Sud 11, Orsay.
\end{itemize}

%\> \>Thesis defended June $11^{th}$, graduated with honors. \\
%\>\>Title: "Search for high mass neutral gauge bosons decaying into a pair \\
%\>\> of muons in the ATLAS detector".\\
%\>\>\\
%\> June 2006: \>Graduation in Particle Physics at the University of Paris-Sud 11, Orsay \\ 
%\>\>Master in Particle Physics with honors. \\
%\> 2003 - 2006:  \>Physics undergraduate at the University of Paris-Sud 11, Orsay.


%%%%%%%%%%%%%%%%%%%%%%%%%%%%%%%%%%%%%%%%%%%%%%%%%%%%%%%%%%%%%%%%%%%%%%%%%%%%%%%%%%%%%%%%%
%\vskip 0.7 cm
%{\bf \large Research Positions}
%\vskip 0.3 cm
%\TabPositions{5.4cm}
%\begin{itemize}
%\itemsep0em
%\item[] Aug. 2015 - Jul. 2020  \tab : Research LD Staff at CERN, Geneva, Switzerland.
%\item[] Feb. 2013 - Jul. 2015  \tab : Research fellow at CERN, Geneva, Switzerland.
%\item[] Nov. 2009 - Jan. 2013  \tab : Research associate at IFAE, Barcelona, Spain.
%\item[] Oct. 2006 - Oct. 2009  \tab : Ph.D. student at CEA-Saclay, France.
%\end{itemize}


%%%%%%%%%%%%%%%%%%%%%%%%%%%%%%%%%%%%%%%%%%%%%%%%%%%%%%%%%%%%%%%%%%%%%%%%%%%%%%%%%%%%%%%%%
\vskip 0.7 cm
{\bf \large Management Roles}
\vskip 0.3 cm
\begin{itemize}
\itemsep0em
\item[] Nov 2017 - present: Top Upgrade team contact (ATLAS)
\item[] Nov 2015 - Mar 2017: Convener of the top-properties physics sub-group (ATLAS)
\item[] Sep 2015 - present: Convener of the FCC software group
\end{itemize}

%%%%%%%%%%%%%%%%%%%%%%%%%%%%%%%%%%%%%%%%%%%%%%%%%%%%%%%%%%%%%%%%%%%%%%%%%%%%%%%%%%%%%%%%%
\vskip 0.7 cm
{\bf \large Other Responsibilities}
\vskip 0.3 cm
\begin{itemize}
\itemsep0em
\item[] Referee for Physics Letters B.
\item[] Member of editorial boards in ATLAS
\item[] Supervision of CERN: fellows, project associate, doctoral and summer students
\end{itemize}

%%%%%%%%%%%%%%%%%%%%%%%%%%%%%%%%%%%%%%%%%%%%%%%%%%%%%%%%%%%%%%%%%%%%%%%%%%%%%%%%%%%%%%%%%
\vskip 0.7 cm
{\bf \large Skills}
\vskip 0.3 cm
\TabPositions{3.2cm}
\begin{itemize}
\itemsep0.3em
\item[] Management : Modern management style and team building strategies;
  Preference for flat hierarchy and responsibility sharing.
\item[] Programming : Excellent knowledge of C, C++, python, shell;
  Packages, such as ROOT, RooFit, RooStat, Geant4, TMVA, pyMC, 
 tensorflow, numpy, matplotlib;
  Code management with SVN, git.
 \item[] Languages : Native French, fluent English, good German, basic Japanese.
 \item[] Outreach : CERN official guide, ATLAS underground escort guide; Open days

\end{itemize}

\newpage

%%%%%%%%%%%%%%%%%%%%%%%%%%%%%%%%%%%%%%%%%%%%%%%%%%%%%%%%%%%%%%%%%%%%%%%%%%%%%%%%%%%%%%%%%
The references cited below can be found in the publication list.
\vskip 0.8 cm
{\bf \Large Selected Research Activities in the ATLAS Experiment}
\vskip 0.4cm

%%%%%%%%%%%%%%%%%%%%%%%%%%%%%%%%%%%%%%%%%%%%%%%%%%%%%%%%%%%%%%%%%%%%%%%%%%%%%%%%%%%%%%%%%
{\bf \underline{2014 - Present : precision top quark physics}}
\vskip 0.3cm
\begin{itemize}[leftmargin=1.3cm]
\itemsep0.8em

%%%%%%%%%%%%%%%%%%%%%%%%%%%%%%%%%%%%%%%%%%%%%
\item[] {\bf Charge asymmetry ($A_C$):}
I was the main contributor to the $t\bar{t}$ charge asymmetry measurement in the semileptonic channel 
using 20.3 fb$^{-1}$ of data at $\sqrt{s}$ = 8~TeV {\bf \color{red} [10]}. I was editor of the paper. This measurement incorporates a number of experimental improvements with respect to the previous analyses and is the most precise one for LHC Run-I. I have been chosen to coordinate the ATLAS $t\bar{t}$ asymmetry effort, composed of three different analyses and about 10 people. I also contributed to the $A_C$ combination with CMS, leading to the first combined differential measurement in the top sector.
\vspace{2.mm}

Using the full Run II dataset at $\sqrt{s}$ = 13~TeV,  I contribute to the ongoing $A_C$ effort as a consultant. I make sure that all the expertise gained previously is properly transferred and applied by the new teams. Thanks to the unfolding method being used, this measurement combines for the fist time at the likelihood level the three independent analyses and should lead to the most precise measurement. 

%%%%%%%%%%%%%%%%%%%%%%%%%%%%%%%%%%%%%%%%%%%%%
\item[] {\bf Top coordination roles: }
From the end of 2015 until early 2017 I accepted to take the responsibility of co-coordinating the top-properties physics sub-group. This was an interesting time when 13~TeV activities started and it was extremely important to finish the 8~TeV contributions as soon as possible. The work consisted in running weekly meetings and closely following the progress of about ten different analyses. I succeeded in significantly helping most of the analyses teams towards analysis approval or publication, and I initiated an overall program for including common EFT limits in FCNC searches.
\vspace{2.mm}

As the future of high energy physics is a subjet of crucial importance, it was a natural choice to accept the invitation to join the top-upgrade team at the end of 2017. The work consisted in organising the upgrade effort towards the publication of the HL/HE-LHC yellow report {\bf \color{red} [19]}. This was a real challenge because all the people that expressed interests in performing an HL or HE-LHC analysis were also deeply involved in the corresponding 13~TeV analysis. In the end, we managed to lead four analyses teams towards publication.

%%%%%%%%%%%%%%%%%%%%%%%%%%%%%%%%%%%%%%%%%%%%%
\item[] {\bf Unfolding: }
For the precision measurements, I developed a novel unfolding tool based on the Fully Bayesian Unfolding technique consisting in the application of Bayes theorem to the problem of unfolding. In this method, a likelihood is used to constrains systematics uncertainties from data and to combine channels, and has been used for several publications in ATLAS already. I am maintaining the package, and a document including detailed examples is under preparation {\bf \color{red} [11]}.

%%%%%%%%%%%%%%%%%%%%%%%%%%%%%%%%%%%%%%%%%%%%%
\item[] {\bf Fragmentation of $b$-quarks:}
A few months ago I started to develop a new analysis aiming at measuring the fragmentation of $b$-quarks in $t\bar{t}$ where one of the $b$-quark from the top decays to a $J/\psi$. Our current generators are tuned to LEP data, where the $b\bar{b}$ pairs are produced as color singlet. The idea of this measurement is to understand if non-singlet $b\bar{b}$ production can play a role in the fragmentation. This measurement could ultimately serve as input for future tuning of generators and help reduce specific types of systematic uncertainties.

%%%%%%%%%%%%%%%%%%%%%%%%%%%%%%%%%%%%%%%%%%%%%
\item[] {\bf Underlying events:}
In collaboration with some colleagues, we started early last year brainstorming about the best way to measure the properties of the underlying events in $t\bar{t}$. Indeed, analyses like the top mass measurement are suffering from a large uncertainty from color re-connection or, more generally, soft QCD uncertainties. The idea is to perform such a measurement in $t\bar{t}$ event to get better tunes for generators and to reduce modeling uncertainties. Currently,  I am responsible for one unfolding technique.


\end{itemize}
\vskip 0.4cm

%%%%%%%%%%%%%%%%%%%%%%%%%%%%%%%%%%%%%%%%%%%%%%%%%%%%%%%%%%%%%%%%%%%%%%%%%%%%%%%%%%%%%%%%%%
{\bf  \underline{2014 - 2017  Muon Spectrometer New Small Wheel}}
\vskip 0.2cm
\begin{itemize}[leftmargin=1.3cm]
\itemsep0.8em
%%%%%%%%%%%%%%%%%%%%%%%%%%%%%%%%%%%%%%%%%%%%%
\item[] 
I participated in the NSW upgrade with a project in which I had taken responsibility for the design, implementation and construction of a system to easily scan Micromegas chambers with X-rays. The device is currently being used to quickly test the uniformity of the response. I also served as a shifter for the quality control of the Micromegas PCBs.
\end{itemize}
\vskip 0.4cm

%%%%%%%%%%%%%%%%%%%%%%%%%%%%%%%%%%%%%%%%%%%%%%%%%%%%%%%%%%%%%%%%%%%%%%%%%%%%%%%%%%%%%%%%%%
{\bf \underline{2010 - 2015 Searches for exotic heavy quarks}}
\vskip 0.2cm
\begin{itemize}[leftmargin=1.3cm]
\itemsep0.8em
%%%%%%%%%%%%%%%%%%%%%%%%%%%%%%%%%%%%%%%%%%%%%
\item[] From the beginning, I played a leading role in the heavy quark searches. In particular, I defined a coherent strategy for vector-like quark searches in ATLAS that allowed these scenarios to be probed in a very sensitive way, and that became the default search strategy. As lead analyzer, I performed the first search at ATLAS for 4$^{th}$ generation up-type quarks $t'\bar{t'} \rightarrow W^{+}bW^{-}\bar{b}$ in the semileptonic final state using 1\,\mathrm{fb}$^{-1}$ of data at a center--of--mass energy of 7\,TeV {\color{red}\bf[4]}.

%%%%%%%%%%%%%%%%%%%%%%%%%%%%%%%%%%%%%%%%%%%%%
\vspace{2.mm}
This was followed by an improved analysis with the full 2011 dataset corresponding to 4.7\,\mathrm{fb}$^{-1}$, in which the search strategy was optimized to further enhance the sensitivity at higher masses by exploiting the characteristic topology of boosted $W$ bosons in the decay of heavy quarks {\color{red}\bf[5]}. 
For this publication, I developed the methods and procedures to include for the first time quasi model-independent limits 
on vector-like quarks (T, B), which were very well received by the theoretical community.


%%%%%%%%%%%%%%%%%%%%%%%%%%%%%%%%%%%%%%%%%%%%%
\vspace{2.mm} 
Using 14.3\,\mathrm{fb}$^{-1}$ of data collected at $\sqrt{s}$ = 8~TeV, I performed the searches for $T\bar{T} \rightarrow Ht+X$ {\color{red}\bf[6]} and $T\bar{T} \rightarrow Wb+X$ {\color{red}\bf[7]} in the 
semileptonic channel and the combination of these searches {\color{red}\bf[7]}. All the procedures and tools I developed 
for those analyses are also used by all the vector-like quark searches at 8~TeV, see for example {\color{red}\bf[8]}.
\vspace{2.mm}

With the full 2012 dataset of 20.3\,\mathrm{fb}$^{-1}$ I finalized the search for $T\bar{T} \rightarrow Wb+X$  for Run-I legacy publication.
This result contains two other analyses: searches for $T\bar{T} \rightarrow Ht+X$
and $B\bar{B} \rightarrow Hb+X$. The combination of $Wb+X$ and $Ht+X$ provided the most stringent constraints on observed lower limits for pair produced heavy up type quarks {\color{red}\bf[9]} from LHC Run-I.  

\end{itemize}
\vskip 0.4cm

 
%%%%%%%%%%%%%%%%%%%%%%%%%%%%%%%%%%%%%%%%%%%%%%%%%%%%%%%%%%%%%%%%%%%%%%%%%%%%%%%%%%%%%%%%%%
{\bf \underline{2010 - 2011 First top measurements}}
\vskip 0.2cm
\begin{itemize}[leftmargin=1.3cm]
\itemsep0.8em
%%%%%%%%%%%%%%%%%%%%%%%%%%%%%%%%%%%%%%%%%%%%%
\item[] I was one of the key people involved in the re-discovery of the top quark at the LHC with the first 7~TeV $pp$ collisions recorded in 2010. I was one of the main contributors to the first cross-section measurement in the semileptonic decay mode based on a fit method measuring simultaneously the $t\bar{t}$ signal and the $W+$jets background. This analysis was included in the first public ATLAS measurement of the top quark production cross section with 2.9\,\mathrm{pb}$^{-1}$ of collision data {\bf \color{red}[2]}.
I participated in updating this analysis with 35\,\mathrm{pb}$^{-1}$ of data, resulting in one of the most precise measurements at that time {\bf \color{red}[3]}.
\end{itemize}
\vskip 0.4cm


%%%%%%%%%%%%%%%%%%%%%%%%%%%%%%%%%%%%%%%%%%%%%%%%%%%%%%%%%%%%%%%%%%%%%%%%%%%%%%%%%%%%%%%%%%
{\bf \underline{2009 - 2012 Tile (Hadronic) Calorimeter}}
\vskip 0.2cm
\begin{itemize}[leftmargin=1.3cm]
\itemsep0.8em
%%%%%%%%%%%%%%%%%%%%%%%%%%%%%%%%%%%%%%%%%%%%%
\item[] 
I have developed a tool based on multivariate analysis for noise suppression in the Hadronic Calorimeter in the detector reconstruction framework and a Data Quality Monitoring tool (via web interface). I have participated in the ATLAS data taking operations as Tile Calorimeter shifter, Data Quality Validator and Data Quality Leader.
\end{itemize}
\vskip 0.4cm


{\bf \underline{2006 - 2009 Muon Spectrometer (MS) and Heavy di-muon resonance}}
\vskip 0.2cm
\begin{itemize}[leftmargin=1.3cm]
\itemsep0.8em
%%%%%%%%%%%%%%%%%%%%%%%%%%%%%%%%%%%%%%%%%%%%%
 \item[] 
As part of my thesis, I studied and initiated a novel triggering logic for the muon system in the absence of toroidal magnetic field. This logic was used during collision runs in this configuration. I also estimated the running time necessary to determine with a given precision the reference geometry of the barrel MS with straight tracks, an estimation used to request collision data without toroidal field. For example, at an instantaneous luminosity of $10^{32}$ cm$^{-2}$s$^{-1}$, 10 hours of collisions are needed to reach an alignment precision of 30 microns. 


%%%%%%%%%%%%%%%%%%%%%%%%%%%%%%%%%%%%%%%%%%%%%
\vspace{2.mm}
I also studied the discovery potential of heavy neutral gauge bosons $Z^{\prime}$ decaying into dimuon pairs. 
Since the MS momentum is dominated by the misalignment for high $p_T$ muons, I have participated in detailed studies to assess the impact of this effect on $Z^{\prime}$ search sensitivity {\bf \color{red}[1]}. For this purpose I developed a sophisticated statistical method which was adopted by the exotics physics group. I also derived a parametrization of the MS misalignment in the detector simulation, and made it available to others as an official tool.

%%%%%%%%%%%%%%%%%%%%%%%%%%%%%%%%%%%%%%%%%%%%%
\vspace{2.mm}
Finally, I participated in hardware activities in the ATLAS cavern with the commissioning of the optical alignment system of the central part of the MS.
\end{itemize}


%%%%%%%%%%%%%%%%%%%%%%%%%%%%%%%%%%%%%%%%%%%%%%%%%%%%%%%%%%%%%%%%%%%%%%%%%%%%%%%%%%%%%%%%%%
\vskip 0.8 cm
{\bf \Large Selected Research Activities in the FCC study group}
\vskip 0.4cm

\begin{itemize}[leftmargin=1.3cm]
\itemsep0.8em

\item[] Since end of 2013, CERN has been undertaking a full design study for post-LHC particle accelerator options in a global context. This was a unique opportunity to design a new project from the beginning and I dedicated a significant fraction of my time to the project, leading to significant contributions to 3 of the 4 published Conceptual Design Reports {\bf \color{red}[12,13,14]}. The three main domains where I contributed are listed below. No date is given as the activity is still ongoing for the most part.

%%%%%%%%%%%%%%%%%%%%%%%%%%%%%%%%%%%%%%%%%%%%%
\item[] {\bf \underline{Software infrastructure}}
\vskip 0.2cm
One of the key aspects of this new project is to allow newcomers to easily contribute. This is the main reason why starting to develop the software infrastructure at the very beginning is extremely important. The software infrastructure is common to the various branches of FCC (hh, ee, eh) so it is requested to be robust and flexible. I am coordinating the activities as software convener. I am also one of the main developer of the framework used for the physics analyses and simulation studies published in the CDRs, including a simple but efficient production system that I developed and maintain. I am also in charge of managing the resources in terms of CPU and disk space.


%%%%%%%%%%%%%%%%%%%%%%%%%%%%%%%%%%%%%%%%%%%%%
\item[] {\bf \underline{Detector Simulation}}
\vskip 0.2cm
In relation with the software infrastructure, I am working on the FCC-hh detector simulation. 
As a starting point I defined the requirements for a hadronic calorimeter in terms of depth and material, as its radius is one of the parameters driving the size of the detector and thus the cost {\bf \color{red}[15]}. I am co-coordinating a group that assess the performance requirements of a calorimeter concept based on ATLAS technologies applied to FCC-hh in full simulation. The concept achieves the needs in terms of performance and is documented in the FCC Vol. 3 CDR {\bf \color{red}[13]}. I have also initiated a study to estimate the impact of the missing gluon-jet emissions in the hadronic interactions modelled by the hadronic string models of Geant4. Work is ongoing and a publication is planned afterwards {\bf \color{red}[16]}.


%%%%%%%%%%%%%%%%%%%%%%%%%%%%%%%%%%%%%%%%%%%%%
\item[] {\bf \underline{Physics at a 100~TeV collider}}
\vskip 0.2cm
I played a leading role in the estimation of the physics reach for an FCC-hh detector. In particular, I estimated the discovery reach and exclusion potential of heavy resonances decaying to di-leptons, di-bosons and di-tops. The results are documented in the CDRs {\bf \color{red}[12,13]}. My work has also been used to estimate the sensitivity of FCC-hh to Higgs couplings (including self-coupling), super-symetric top partners and flavour-changing neutral current top decays {\bf \color{red}[12,13]}.

Given that CERN is also considering an energy upgrade of the LHC, the so called HE-LHC with a center of mass energy of 27\,TeV, I also derived the expected reach for heavy resonances in that context. This is documented in the HE-LHC CDR {\bf \color{red}[14]} and in the HL/HE-LHC report {\bf \color{red}[18]}. I am also one of the main contributors to a study that aims at discriminating $Z'$ at HE-LHC in case of an evidence/discovery after the HL-LHC {\bf \color{red}[17, 18]}. 

 \vspace{2.5mm}





\end{itemize}




%%%%%%%%%%%%%%%%%%%%%%%%%%%%%%%%%%%%%%%%%%%%%%%%%%%%%%%%%%%%%%%%%%%%%%%%%%%%%%%%%%%%%%%%%%
\vskip 0.8 cm
{\bf \large References}
\vskip 0.2cm
\begin{itemize}[leftmargin=1.3cm]
\item[] The people listed here are certainly willing to be contacted and/or send a written recommendation. I indicated the approximate date at which these people stopped following my work. Please get in touch with me to arrange for references to be sent or to provide you with contact information.
\end{itemize}
\newpage

%\hskip 0.1cm
\begin{tabbing}
\itemsep0.8em
 \hskip 1.3cm  \=  \hskip 6.cm \=  \hskip 1.5cm  \= \hskip 6.cm \=\\ 
\> {\bf Martin Aleksa}   \> - \> {\bf Daniel Froidevaux}  \> 2018    \\
\> {\bf Tancredi Carli}  \> - \> {\bf Paolo Iengo}   \> 2016\\
\> {\bf Richard Hawkings}  \> - \> {\bf Aurelio Juste}   \> 2015\\
\> {\bf Ana Henriques}  \> - \> {\bf Martine Bosman}   \> 2013\\
\> {\bf Michelangelo Mangano} \> - \> {\bf Claude Guyot}  \> 2009 \\
\> {\bf Patrick Janot}  \> - \\
\> {\bf Werner Riegler}   \> - \\
%\> {\bf Daniel Froidevaux}  \> 2018   \\
%\> {\bf Paolo Iengo}   \> 2016 \\
%\> {\bf Aurelio Juste}   \> 2015 \\
%\> {\bf Martine Bosman}   \> 2013 \\
%\> {\bf Claude Guyot}  \> 2009 

\end{tabbing}


%%%%%%%%%%%%%%%%%%%%%%%%%%%%%%%%%%%%%%%%%%%%%%%%%
%Talks at CONFERENCES
%%%%%%%%%%%%%%%%%%%%%%%%%%%%%%%%%%%%%%%%%%%%%%%%%
\vskip 0.8 cm
{\bf  \large  Talks at international conferences}
\vskip 0.3 cm


\begin{tabbing}
\itemsep0.8em
 \hskip  1.cm  \= \hskip  3.3cm  \=  \hskip 7.3cm \=   \\ 
 \> {\bf Nov 2018 } \> Higgs Couplings, Tokyo  \> \\
 \> \> {\bf Higgs self-couplings at FCC-hh} \>  \\[0.2cm]
 
 \>{\bf Jul 2018 } \> ICHEP, Seoul \>  \\
 \> \>  {\bf Heavy Resonance searches at FCC-hh}\>  \\[0.2cm]
 
 \>{\bf Jul 2018 } \> ICHEP, Seoul \>  \\
 \> \>  {\bf Top physics at FCC}\>  \\[0.2cm]
 
 \>{\bf May 2018 } \> $14^{th}$ PISA meeting, Elba \>  \\
  \>  \> {\bf Design and performance studies of the  calorimeter system } \> \\
  \>  \> {\bf for an FCC-hh experiment}\>  \\[0.2cm]
  
 \> {\bf August 2013 } \> Rencontres du Vietnam: Windows on the Universe, Quy Nhon \>  \\
  \>  \> {\bf Top quark production and top quark properties at the LHC} \\[0.2cm]

 \> {\bf  May 2011} \> Flavor Physics and CP Violation, Kibbutz Maale Hachamisha \>  \\
  \>  \> {\bf Top physics with the ATLAS detector} \> \\[0.2cm]
\end{tabbing}


%%%%%%%%%%%%%%%%%%%%%%%%%%%%%%%%%%%%%%%%%%%%%%%%%
%Talks at WORKSHOP
%%%%%%%%%%%%%%%%%%%%%%%%%%%%%%%%%%%%%%%%%%%%%%%%%
\vskip 0.8 cm
{\bf  \large  Talks at workshops}
\vskip 0.3 cm
\begin{tabbing}
\itemsep0.8em
 \hskip  1.cm  \= \hskip  3.3cm  \=  \hskip 7.3cm \=   \\ 
  \> {\bf Nov 2018} \> Faculty meeting, CERN\>  \\
  \> \>  {\bf Searching for New Physics at future colliders:}\>  \\
  \> \>  {\bf indirect versus direct reach} \> \\[0.2cm]

  \> {\bf Oct 2018} \> Workshop on high-energy implications of flavor anomalies, CERN\>  \\
  \> \>  {\bf High mass di-muon interpretations } \>\\[0.2cm]

  \> {\bf Apr 2018} \> FCC week, Amsterdam\>  \\
  \> \>  {\bf Heavy resonances at 100TeV } \>\\[0.2cm]

  \> {\bf Apr 2018} \> FCC week, Amsterdam\>  \\
  \> \>  {\bf FCC-hh analysis chain } \> \\[0.2cm]

  \> {\bf Jan 2018} \> $2^{nd}$ FCC physics workshop, CERN\>  \\
  \> \>  {\bf Discovery reach for heavy resonances at 100 TeV } \>\\[0.2cm]

  \> {\bf Nov 2016} \> LHC TOP WG meeting, CERN\>  \\
  \> \>  {\bf   LHC results on EFTs from top quark measurements} \>\\[0.2cm]

  \> {\bf Apr 2016} \> FCC week, Rome\>  \\
  \> \>  {\bf Parameterized simulation and analysis  } \> \\[0.2cm]

  \> {\bf Apr 2016} \> FCC week, Rome\>  \\
  \> \>  {\bf Performance and potential of an ATLAS-like HCal Tile for FCC } \> \\[0.2cm]

  \> {\bf Feb 2015} \> Workshop for future detector technologies in view of FCC-hh, CERN \>  \\
  \> \>  {\bf Performance requirements of the ECal and HCal} \> \\[0.2cm]
   
  \> {\bf Sep 2014} \> Workshop on Vector-like Quarks, DESY\>  \\
  \> \>  {\bf Interpretation of Vector-like Quarks experimental results} \> \\[0.2cm]

  \> {\bf Sep 2014} \> Workshop on Vector-like Quarks, DESY\>  \\
  \> \>  {\bf Searches in single-lepton final states at ATLAS} \> \\[0.2cm]

  \> {\bf Jun 2014} \>GDR-Terascale workshop, LLR Palaiseau \>  \\
  \> \>  {\bf Introduction to the FCC project} \> \\[0.2cm]

  \> {\bf Jun 2014} \>  $1^{st}$ Future Hadron Collider Workshop, CERN\>  \\
  \> \>  {\bf First considerations about hadronic calorimetry} \> \\[0.2cm]

  \> {\bf Jun 2014} \>  $1^{st}$ Future Hadron Collider Workshop, CERN \>  \\
  \> \>  {\bf Progress report on DD4HEP/Geant4 based simulation} \> \\[0.2cm]

  \> {\bf Mar 2013} \> Top LHC-France workshop, Lyon \>  \\
  \> \>  {\bf Using profile likelihood ratios at the LHC} \> \\[0.2cm]

  \> {\bf Mar 2012} \> LPCC, Implications of LHC results for TeV-scale physics, CERN \>  \\
  \> \>  {\bf ATLAS Exotic top and fourth generation searches} \> \\[0.2cm]

  \> {\bf Mar 2012} \> $4^{th}$ fermion generation and single-top production Workshop, Leinsweiler Plafz \>  \\
  \> \>  {\bf Searches for fourth generation fermions at ATLAS} \> \\[0.2cm]

  \> {\bf Oct 2011} \> GDR-Terascale workshop, Marseille \>  \\
  \> \>  {\bf $4^{th}$ generation searches in ATLAS and CMS} \> \\[0.2cm]

  \> {\bf Sep 2007} \> Physic ATLAS France workshop, Biarritz \>  \\
  \> \>  {\bf Search for high mass resonances in the dimuon channel. } \> \\
  \> \>  {\bf Study of the impact of muon spectrometer alignment} \> \\[0.2cm]
  
  \> {\bf Dec 2006} \> Muon Barrel Cosmic Run and Commissioning Workshop, CERN \>  \\
  \> \>  {\bf Alignment of the Muon spectrometer using the first} \> \\
  \> \>  {\bf cosmic straight tracks} \> \\[0.2cm]

\end{tabbing}


%%%%%%%%%%%%%%%%%%%%%%%%%%%%%%%%%%%%%%%%%%%%%%%%%
%Talks at Seminars
%%%%%%%%%%%%%%%%%%%%%%%%%%%%%%%%%%%%%%%%%%%%%%%%%
\vskip 0.8 cm
{\bf  \large  Seminars}
\vskip 0.3 cm
\begin{tabbing}
\itemsep0.8em
 \hskip  1.cm  \= \hskip  3.3cm  \=  \hskip 7.3cm \=   \\ 
 
  \> {\bf Apr 2012} \> KEK, Tsukuba \>  \\
  \> \>  {\bf $4^{th}$ generation and heavy quark at LHC}\>  \\[0.2cm]

  \> {\bf Feb 2012} \> LPSC, Grenoble\>  \\
  \> \>  {\bf $4^{th}$ generation and heavy quark at LHC}\>  \\[0.2cm]

  \> {\bf Jan 2012} \> LAPP, Annecy\>  \\
  \> \>  {\bf $4^{th}$ generation and heavy quark at LHC}\>  \\[0.2cm]

  \> {\bf Jan 2012} \> LPC, Clermont-Ferrand\>  \\
  \> \>  {\bf $4^{th}$ generation and heavy quark at LHC}\>  \\[0.2cm]

  \> {\bf Sep 2009} \> IFAE, Barcelona\>  \\
  \> \>  {\bf Search for high mass resonances in the dimuon channel}\>  \\
  \> \>  {\bf with the ATLAS experiment}\>  \\[0.2cm]

\end{tabbing}


%%%%%%%%%%%%%%%%%%%%%%%%%%%%%%%%%%%%%%%%%%%%%%%%%
%Talks at SEMINARS
%%%%%%%%%%%%%%%%%%%%%%%%%%%%%%%%%%%%%%%%%%%%%%%%%
%\vskip 0.8 cm
%{\bf  \large  Seminars}

%\vskip 0.3 cm

%\begin{enumerate} 

%\item "$4^{th}$ generation and heavy quark at LHC"\\
%High Energy Accelerator Research Organization (KEK)\\
%Tsukuba, Japan, April 2012.

%\item "$4^{th}$ generation and heavy quark at LHC"\\
%Laboratoire de Physique Subatomique et de Cosmologie (LPSC)\\
%Grenoble, France, February 2012.

%\item "$4^{th}$ generation and heavy quark at LHC"\\
%Laboratoire d'Annecy le Vieux de Physique des Particules (LAPP)\\
%Annecy, France, January 2012.

%\item "$4^{th}$ generation and heavy quark at LHC"\\
%Laboratoire de Physique Corpusculaire (LPC)\\
%Clermont-Ferrand, France, January 2012.

%\item "Search for high mass resonances in the dimuon channel with the ATLAS experiment"\\
%Institut de F\'{\i}sica d'Altes Energies (IFAE)\\
%Barcelona, Spain, September 2009

%\end{enumerate}

%%%%%%%%%%%%%%%%%%%%%%%%%%%%%%%%%%%%%%%%%%%%%%%%%
%Talks at CONFERENCES
%%%%%%%%%%%%%%%%%%%%%%%%%%%%%%%%%%%%%%%%%%%%%%%%%
%\begin{enumerate} 
%\item "Higgs self-couplings at FCC-hh"\\
%Higgs couplings 2018\\
%Tokyo, Japan, November 2018

%\item "Heavy Resonance searches at FCC-hh"\\
%ICHEP 2018\\
%Seoul, South Korea, July 2018

%\item "Top physics at FCC"\\
%ICHEP 2018\\
%Seoul, South Korea, July 2018

%\item "Design and performance studies of the calorimeter system for an FCC-hh experiment"\\
%14th PISA meeting on advanced detectors\\
%Isola d'Elba, Italy, May 2018

%\item "Top quark production and top quark properties at the LHC"\\
%Inaugural Conference Windows on the Universe Rencontres du Vietnam \\
%Quy Nhon, Vietnam, August 2013.

%\item "Top physics with the ATLAS detector"\\
%Flavor Physics and CP Violation\\ 
%Kibbutz Maale Hachamisha, Israel, May 2011

%\end{enumerate}

%%%%%%%%%%%%%%%%%%%%%%%%%%%%%%%%%%%%%%%%%%%%%%%%%
%Talks at WORKSHOP
%%%%%%%%%%%%%%%%%%%%%%%%%%%%%%%%%%%%%%%%%%%%%%%%%
%\vskip 0.8 cm
%{\bf  \large  Talks at Workshops}
%\vskip 0.3 cm
%\begin{enumerate} 
%\item "Searching for New Physics at future colliders: indirect versus direct reach"\\
%CERN Faculty meeting\\
%CERN, Switzerland, November 2018

%\item "High mass di-muon interpretations"\\
%Workshop on high-energy implications of flavor anomalies\\
%CERN, Switzerland, October 2018

%\item "Heavy resonances at 100TeV"\\
%FCC week 2018\\
%Amsterdam, Netherlands, April 2018

%\item "FCC-hh analysis chain"\\
%FCC week 2018\\
%Amsterdam, Netherlands, April 2018

%\item "Parametrized simulation and analysis"\\
%FCC week 2016\\
%Rome, Italy, April 2016

%\item "Performance and potential of an ATLAS-like HCal Tile for FCC"\\
%FCC week 2016\\
%Rome, Italy, April 2016

%\item "Performance requirements of the electromagnetic and hadronic calorimeters"\\
%Workshop on requirements for future detector technologies in view of FCC-hh\\
%CERN, Switzerland, February 2015

%\item "Interpretation of Vector-like Quarks experimental results"\\
%Workshop on Vector-like Quarks\\
%DESY, Germany, September 2014

%\item "Searches in single-lepton final states at ATLAS"\\
%Workshop on Vector-like Quarks 2014 \\
%DESY, Germany, September 2014

%\item "Introduction to the FCC project"\\
%GDR-Terascale workshop\\
% LLR Palaiseau, France, June 2014.

%\item "First considerations about hadronic calorimetry"\\
%$1^{st}$ Future Hadron Collider Workshop\\
%CERN, Switzerland, June 2014.

%\item "Progress report on DD4HEP/Geant4 based simulation"\\
%$1^{st}$ Future Hadron Collider Workshop\\
%CERN, Switzerland, June 2014.

%\item "Using profile likelihood ratios at the LHC"\\
%Top LHC-France workshop \\
%Lyon, France, March 2013.

%\item "ATLAS Exotic top and fourth generation searches"\\
%LPCC, Implications of LHC results for TeV-scale physics\\
%CERN, Switzerland, March 2012.

%\item "Searches for fourth generation fermions at ATLAS"\\
%Fourth fermion generation and single-top production Workshop\\
%Leinsweiler Plafz, Germany, March 2012.

%\item "$4^{th}$ generation searches in ATLAS and CMS"\\
%GDR-Terascale workshop\\
%CPPM, Marseille, France, October 2011

%\item "Search for high mass resonances in the dimuon channel. Study of the impact of muon spectrometer alignment"\\
%Physic ATLAS France workshop\\
% Biarritz, France, September 2007.	

%\item "Alignment of the Muon spectrometer using the first cosmic straight tracks"\\
%Muon Barrel Cosmic Run and Commissioning Workshop\\
%CERN, Switzerland, December 2006.


%end{enumerate}

\end{document}
