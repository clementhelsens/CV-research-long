\documentclass[12pt]{article}
\usepackage[francais]{babel}
\usepackage[latin1]{inputenc}
\usepackage{fancyhdr}
\usepackage{bibunits}
\usepackage{gb4e}
\usepackage{enumitem}
\usepackage{tabto}
\pagestyle{empty}
\oddsidemargin 0.0in
\textwidth 6.5in
\topmargin -0.75in
\textheight 9.5in

\renewcommand{\labelenumi}{\arabic{enumi})}

\begin{document}
\pagestyle{fancy}
\fancyhf{}


%%%%%%%%%%%%%%%%%%%%%%%%%%%%%%%%%%%%%%%%%%%%%%%%%%%%%%%%%%%%%%%%%%%%%%%%%%%%%%%%%%%%%%%%%
\begin{center}
\vskip 1cm
{\bf \Large Curriculum Vitae: Cl\'ement Helsens} \\
\end{center}
\vskip 0.5cm
\hskip 14 cm January 2019
\vskip 1.8cm

%%%%%%%%%%%%%%%%%%%%%%%%%%%%%%%%%%%%%%%%%%%%%%%%%%%%%%%%%%%%%%%%%%%%%%%%%%%%%%%%%%%%%%%%%
{\bf \large Personal Data}
\vskip 0.3 cm
\TabPositions{2.5cm}
\begin{itemize}
\itemsep0em
%\item[] Name \tab : Helsens Cl\'ement
\item[] Birthdate  \tab : September 17, 1982
\item[] Birthplace \tab :   Roubaix (France) 
\item[] Citizenship \tab :  French
\item[] Civil status \tab : Married, 2 children
\item[] Address \tab : CERN, CH-1211 Gen\`eve 23, Suisse
\item[] Phone \tab : +41 22 767 1165  / +33 6 78 79 64 56
\item[] E-mail \tab : clement.helsens@cern.ch
\end{itemize}


%%%%%%%%%%%%%%%%%%%%%%%%%%%%%%%%%%%%%%%%%%%%%%%%%%%%%%%%%%%%%%%%%%%%%%%%%%%%%%%%%%%%%%%%%
\vskip 0.7 cm
{\bf \large Education and Research Positions}
\vskip 0.3 cm
\TabPositions{2.5cm}
\begin{itemize}
\itemsep0em
\item[] Aug 2015 - Jul 2020: Research LD Staff at CERN, Geneva, Switzerland.
\item[] Feb 2013 - Jul 2015: Research fellow at CERN, Geneva, Switzerland.
\item[] Nov 2009 - Jan 2013: Research associate at IFAE, Barcelona, Spain.
\item[] Oct 2006 - Oct 2009: Ph.D. student at CEA-Saclay, France.
%\item[] 2006 - 2009: Ph.D. thesis at the CEA-Saclay and University of Paris-Sud 11, Orsay. 
%\tab \tab Supervisors C. Guyot and H. Bachacou. \\
%\tab \tab "Search for high mass neutral gauge bosons decaying into a pair of muons\\
%\tab \tab in the ATLAS detector".
%\item[]June 2006: Master of Science at the University of Paris-Sud 11, Orsay%, \\
%\tab \tab Particle Physics
%\item[] 2003 - 2006: Bachelor in Physics at the University of Paris-Sud 11, Orsay.
\end{itemize}

%\> \>Thesis defended June $11^{th}$, graduated with honors. \\
%\>\>Title: "Search for high mass neutral gauge bosons decaying into a pair \\
%\>\> of muons in the ATLAS detector".\\
%\>\>\\
%\> June 2006: \>Graduation in Particle Physics at the University of Paris-Sud 11, Orsay \\ 
%\>\>Master in Particle Physics with honors. \\
%\> 2003 - 2006:  \>Physics undergraduate at the University of Paris-Sud 11, Orsay.


%%%%%%%%%%%%%%%%%%%%%%%%%%%%%%%%%%%%%%%%%%%%%%%%%%%%%%%%%%%%%%%%%%%%%%%%%%%%%%%%%%%%%%%%%
%\vskip 0.7 cm
%{\bf \large Research Positions}
%\vskip 0.3 cm
%\TabPositions{5.4cm}
%\begin{itemize}
%\itemsep0em
%\item[] Aug. 2015 - Jul. 2020  \tab : Research LD Staff at CERN, Geneva, Switzerland.
%\item[] Feb. 2013 - Jul. 2015  \tab : Research fellow at CERN, Geneva, Switzerland.
%\item[] Nov. 2009 - Jan. 2013  \tab : Research associate at IFAE, Barcelona, Spain.
%\item[] Oct. 2006 - Oct. 2009  \tab : Ph.D. student at CEA-Saclay, France.
%\end{itemize}


%%%%%%%%%%%%%%%%%%%%%%%%%%%%%%%%%%%%%%%%%%%%%%%%%%%%%%%%%%%%%%%%%%%%%%%%%%%%%%%%%%%%%%%%%
\vskip 0.7 cm
{\bf \large Management roles}
\vskip 0.3 cm
\begin{itemize}
\itemsep0em
\item[] Sep 2015 - present: Convener of the FCC software group
\item[] Nov 2017 - present: Top Upgrade team contact
\item[] Nov 2015 - Mar 2017: Convener of the top-properties physics sub-group
\end{itemize}

%%%%%%%%%%%%%%%%%%%%%%%%%%%%%%%%%%%%%%%%%%%%%%%%%%%%%%%%%%%%%%%%%%%%%%%%%%%%%%%%%%%%%%%%%
\vskip 0.7 cm
{\bf \large Other responsibilities}
\vskip 0.3 cm
\begin{itemize}
\itemsep0em
\item[] Referee for Physics Letters B.
\item[] Member of editorial boards in ATLAS
\end{itemize}

%%%%%%%%%%%%%%%%%%%%%%%%%%%%%%%%%%%%%%%%%%%%%%%%%%%%%%%%%%%%%%%%%%%%%%%%%%%%%%%%%%%%%%%%%
\vskip 0.7 cm
{\bf \large Skills}
\vskip 0.3 cm
\TabPositions{3.2cm}
\begin{itemize}
\itemsep0.3em
\item[] Management : Modern management style and team building strategies;
  Preference for flat hierarchy and responsibility sharing.
\item[] Programming : Very good knowledge of C, C++, Python, xml, json, shell;
  Packages, such as ROOT, RooFit, RooStat, Geant4, TMVA, pyMC, 
 tensorflow, numpy, matplotlib;
  Code management with SVN, Git.
 \item[] Languages : Native French, fluent English, good German, basic Japanese.
 \item[] Outreach : CERN official guide, ATLAS underground escort guide; Open days

\end{itemize}

\newpage

%%%%%%%%%%%%%%%%%%%%%%%%%%%%%%%%%%%%%%%%%%%%%%%%%%%%%%%%%%%%%%%%%%%%%%%%%%%%%%%%%%%%%%%%%
Links to the references given below can be found in the publication list.
\vskip 0.8 cm

{\bf \large Selected Research Activities in the ATLAS Experiment}
\vskip 0.2cm
%\vskip 0.2cm
%\hskip 0.1cm {\bf As member of the ATLAS Experiment (since October 2006)}\\
\begin{itemize}[leftmargin=1.3cm]
\itemsep0.8em

 \item[] 2014 - 2017 {\bf Muon Spectrometer New Small Wheel}\\
 \item[] 2010 - 2014 {\bf Searches for exotic heavy quarks}\\
 \item[] 2010 - Present {\bf Precision top quark physics}\\
 \item[] 2006 - 2009 {\bf Muon Spectrometer and Heavy di-muon resonance}\\


 \item[] {\bf Muon Spectrometer and Heavy di-muon resonance}\\
One of the main challenges for the good performance of the Muon Spectrometer (MS) is to determine the position of the muon chambers. 
It can be obtained using straight tracks from cosmic or collision data without toroidal magnetic field. 
When taking collision data without toroidal field, all the straight tracks will be reconstructed in the MS as high $p_T$ muons, as only the multiple 
scattering will play a role in smearing their trajectory. In addition, the first trigger level logic based on measuring the deviation in the magnetic field,
could be saturated by low $p_T$ muons.
\vspace{2.5mm}

As part of my thesis, I studied and initiated a novel triggering logic for the muon system in the absence of toroidal magnetic field, logic used during 
collision runs in this configuration. I also estimated the running time necessary to determine with a given precision the reference geometry 
of the barrel MS with straight tracks, estimation used to request collision data without toroidal field. 
As an example at an instantaneous luminosity of $10^{-32}$ cm$^{-2}$s$^{-1}$, 10 hours of collisions are needed to reach an alignment precision of 30 microns. 
\vspace{2.5mm}

 
During my thesis I also studied the discovery potential of heavy neutral gauge bosons $Z^{\prime}$ decaying into dimuon pairs. 
As for muons of high momentum, the MS momentum resolution is dominated by the misalignment, I have participated in detailed studies to assess the impact of this effect on $Z^{\prime}$ 
search sensitivity [1]. For this purpose I developed a sophisticated statistical method which was adopted by the exotics physics group. I also derived a parametrization of the MS misalignment 
in the detector simulation, and made it available to others as an official tool.
Finally, I participated in hardware activities in the ATLAS cavern with the 
commissioning of the optical alignment system of the central part of the MS.


\item[] {\bf Muon Spectrometer New Small Wheel}\\
In order to maintain the excellent performance of the MS at higher instantaneous luminosity, the forward part called the New Small Wheel (NSW)
will be replaced with micro pattern gaseous detector called Micromegas.
I participated in the NSW upgrade with a project where I have taken responsibility for the design, implementation and construction of a system to easily scan Micromegas 
chambers with X-rays. The device is currently being used to quickly test the uniformity of the response.

\vspace{2.5mm}


\item[] {\bf Precision top quark physics}\\

The precise measurement of the $t\bar{t}$ production cross section provides an important test of perturbative 
QCD calculations and a probe for new physics in the top quark sector, such as non-standard top quark 
production mechanisms and/or decay modes. 
During my post-doc at IFAE I was one of the key people involved in the re-discovery of the top quark at the LHC with the first 7~TeV $pp$ collisions recorded in 2010. I was one of the main contributors to the first cross-section measurement in the 
semileptonic decay mode based on a fit method measuring simultaneously the $t\bar{t}$ signal and the $W$+jets background,
relying mostly 
on the shape information contained in the invariant mass of the three jets forming the hadronically decaying top quark. 
This analysis was included in the first public ATLAS measurement of the top quark production cross section with 2.9~pb$^{-1}$ of collision data [2].
%which was published in Eur. Phys. J. C. 
I participated in updating this analysis with 35~pb^{-1} of data, resulting in one of the most precise measurements at the time [3].

13TeV: AC, fragmentation, underlying events, spin density
Top-properties sub-group convener
Uprade contact for top-group

\vspace{2.5mm}

The observation of an unexpectedly large forward-backward (FB) asymmetry in $t\bar{t}$ production by the Tevatron experiments constitutes one 
of the most tantalizing hints of new physics in the top quark sector. 
%The latest inclusive values reported by the CDF and DØ Collaborations are around 
%two standard deviations above the SM predictions and even larger departures are found for other related measurements. With the end of data taking 
%at the Tevatron and the large top quark samples being collected by the ATLAS experiment, there is the exciting possibility of a precise measurement at 
%the LHC that could possibly indicate whether or not the Tevatron deviations are real. 
%At the LHC, despite the charge-symmetric initial state, it is possible to 
%define a charge asymmetry sensitive to the same underlying dynamics as the FB asymmetry at the Tevatron. However, this is a challenging 
%This is a challenging measurement since the expected asymmetry from new physics at the LHC would be quite small, at the few percent level, which requires keeping 
%systematic uncertainties below 1\%. 
At the LHC this is a challenging measurement since the expected asymmetry from new physics would be at the few percent level, which requires keeping 
systematic uncertainties below 1\%. 
I am currently one of the main contributors to the $t\bar{t}$ charge asymmetry measurement in the semileptonic channel 
using 20.3 fb$^{-1}$ of data at $\sqrt{s}$ = 8~TeV. This measurement incorporates a number of experimental improvements with respect to the previous 
analyses and should be the most precise one for LHC Run I.  The analysis is being internally reviewed and a preliminary result is planned for the Top@20 workshop.
I am currently coordinating the final steps and editing the paper [10]. 
For this analysis, together with a student, I developed a novel unfolding tool based on the Fully Bayesian Unfolding technique consisting in the application of 
Bayes theorem to the problem of unfolding. We also plan to publish a paper documenting this tool after the analysis has been published.
Since end of November 2014 I have been chosen to coordinate the ATLAS $t\bar{t}$ asymmetry effort, composed of three 
different analyses and about 10 persons. 


\item[] {\bf Searches for exotic heavy quarks}\\
Many new physics models aimed at addressing some of the limitations of the SM involve the presence of exotic quarks, heavier than the top quark. 
%For example, models with a fourth generation of heavy chiral fermions, including a $t'$ and a $b'$ quark, could provide new sources of CP violation 
%(4x4 CKM matrix) to explain the matter-antimatter asymmetry in the universe. 
As the simplest fourth generation models are now disfavoured by experiments, 
attention has turned to another possibility involving weak-isospin singlets or doublets of vector-like quarks,
which appear in many extensions of the SM such as Little Higgs or extra-dimensional models. In these models a top-partner quark, 
for simplicity here referred to as $t'$, often plays a key role in canceling the quadratic divergences in the Higgs boson mass induced by radiative corrections 
involving the top quark. The high center-of-mass energy and integrated luminosity in $pp$ collisions available at the LHC offers a unique opportunity to probe 
these scenarios. At the LHC, these new heavy quarks would be predominantly produced in pairs via the strong interaction for masses below $\sim$1~TeV. 
In the case of 4$^{th}$ generation models, the heavy up-type quark ($t'$) would decay dominantly like the top quark, i.e. $t' \rightarrow Wb$, leading to the same 
final state signatures as top quark pair production. In the case of vector-like quarks (T), neutral current decay modes such as $T \rightarrow Zt$ and 
$T \rightarrow  Ht$ can compete with, or even dominate over $T \rightarrow Wb$ decays, leading to spectacular final state signatures with e.g. at 
least 6 jets and at least 4 b-tagged jets.
\vspace{2.5mm}

Since 2010, I am playing a leading role in the program of heavy quark searches in ATLAS being an analysis contact person and coordinating a group of 
approximately 8 people. 
\vspace{2.5mm}

As lead analyzer, I performed the first search at ATLAS for 4$^{th}$ generation up type quarks
$t'\bar{t'} \rightarrow W^{+}bW^{-}\bar{b}$ in the semileptonic final state using 1~fb^{-1} of data at
a center of mass energy of 7~TeV [4].
\vspace{2.5mm}
%This analysis set a lower limit of $m_{t'} >$ 404~GeV at 95\% C.L. [4]. 

This was followed by an improved analysis with the full 
2011 dataset, corresponding to 4.7~fb^{-1}, where the search strategy was optimized to further enhance the sensitivity at higher masses by exploiting 
the characteristic topology of boosted $W$ bosons in the decay of heavy quarks [5]. 
%This analysis set a lower limit of $m_{t'} >$  656 GeV at 95\% C.L. [5].
For this publication, I developed the methods and procedures to include for the first time quasi-model-independent limits 
on vector-like quarks, a more appealing possibility since the discovery of a Higgs-like boson at the LHC, which were very well received by the theoretical community.
I have played a leading role in defining a coherent strategy for vector-like quark searches in ATLAS that should allow to probe these scenarios in a very sensitive way.
\vspace{2.5mm}
%This involves designing a set of searches able to probe the two dimensional plane of BR($t' \rightarrow Wb$) vs BR($t' \rightarrow Ht$) as a function of $m_{t'}$. 

Using 14.3~fb$^{-1}$ of data collected at $\sqrt{s}$ = 8~TeV, I performed the searches for $T\bar{T} \rightarrow Ht+X$ [6] and $T\bar{T} \rightarrow Wb+X$ [7] in the 
semileptonic channel and the combination of these searches [7]. All the procedures and tools I developed 
for those analyses are also used by all the vector-like quark searches at 8~TeV, see for example [8].
\vspace{2.5mm}

With the full 2012 dataset of 20.3~fb$^{-1}$ I finalized the search for $T\bar{T} \rightarrow Wb+X$  for Run-I legacy publication.
This preliminary result, to be published in JHEP soon after being presented for the first time at the Moriond EW 2105 conference, contains two other analyses: searches for $T\bar{T} \rightarrow Ht+X$
and $B\bar{B} \rightarrow Hb+X$. The combination of $Wb+X$ and $Ht+X$ represents the most stringent constraints to date on observed lower limits for pair produced heavy up type quarks [9].  
%observed lower limits on the heavy $t'$ quark mass range between 730 GeV and 950 GeV representing the most stringent constraints to date.


%\item {\bf Tile (Hadronic) Calorimeter}\\
%I have developed a tool based on multivariate analysis for noise suppression in the 
%Hadronic Calorimeter in the detector reconstruction framework and a Data Quality Monitoring tool (via web interface) in order to monitor and provide 
%automatic checks assessing a simple way to control data quality.
%From January 2010 to December 2012, I have participated in the ATLAS data taking operations as Tile Calorimeter shifter, Data Quality Validator and 
%Data Quality Leader.



\end{itemize}


\vskip 0.5cm
%\hskip 0.5cm {\bf As member of the FCC Collaboration, CERN, Geneva, Switzerland}\\
\hskip 0.1cm {\bf As member of the FCC Collaboration (since October 2013)}\\
\begin{itemize}[leftmargin=1.3cm]
\itemsep0.8em

%\item[] CERN is undertaking a full design study for post-LHC particle accelerator options in a global context. The Future Circular Collider (FCC) study has an emphasis on hadron-hadron (FCC-hh), electron-positron (FCC-ee) and electron-hadron (FCC-eh) high-energy frontier machines. The study is exploring the potential of such colliders in terms of physics reach as well as infrastructure and operation concepts and considering the technology research and development programs that would be required to build a future circular collider. 
%An energy in the center of mass of 100~TeV in a 100~km tunnel is considered for FCC-hh as benchmark.

\item[] {\bf Software infrastructure}\\
One of the key aspects of this new project is to allow newcomers to easily contribute. This is the main reason why starting to develop the software infrastructure 
at the very beginning is extremely important. 
%The software infrastructure should be common to the various branches of FCC (hh, ee, eh) so it is requested be robust and flexible. 
I am participating in brainstorming and testing activities within a task force, aiming at finding the most suitable solutions in terms of detector description, 
data model, simulation, production tool, database, analysis framework. I am also the contact person between the FCC software group and FCC--hh physic groups.



\item[] {\bf Detector Simulation}\\
In relation with the software infrastructure, I am working on the FCC-hh detector simulation using the Geant4 toolkit. 
I am defining the requirements for a hadronic calorimeter in terms of depth, granularity and material, as its radius is one of the parameter driving the size of the detector.
The goal is motivate the fact that not less than 12 nuclear interaction length ($\lambda$) are needed to contain hadronic showers within the calorimeter and ensure 
good high $p_T$  jet and missing transverse energy resolution (constant term). I am in charge of a preliminary study group to assess the performance requirements of the electromagnetic 
and hadronic calorimeters.

\item[] {\bf Physics at a 100~TeV collider}\\
The overall size and cost of the detector will be also driven by the resolution on high $p_T$ muons due to the charged particle bending power proportional to $BL^2$.
It is thus important to understand what would be the impact on the discovery potential for new physics involving high $p_T$ muons for different resolution scenarios. 
I derived such discovery potential using heavy neutral gauge bosons decaying into a pair of muons, $Z^{\prime} \rightarrow \mu^{+}\mu^{-}$ as a function of the momentun resolution 
and those results are currently used for the detector design.
 \vspace{2.5mm}

Being able to tag hyper-boosted top quark ($p_T > 3$~TeV) would be one of the main requirements for a 100~TeV machine as they allow for example to select a pure 
sample of $t\bar{t}H$, and to measure precisely the top anomalous couplings through $W^* \rightarrow t\bar{b}$ at high invariant mass.
I started to work on this topic and I am studying  the possibility to discriminate very high $p_T$ tops from heavy flavour jets.

\end{itemize}







\vskip 0.8 cm
{\bf \large References}
\vskip 0.2cm
\begin{itemize}[leftmargin=1.3cm]
\item[] The people listed here are certainly willing to be contacted and/or send a written recommendation. I indicated the approximate date at which these people stopped following my work.
Please get in touch with me to arrange for references to be sent or to provide you with contact information.
\end{itemize}

%\hskip 0.1cm
\begin{tabbing}
\itemsep0.8em
 \hskip 1.3cm  \=  \hskip 5.3cm \=  \hskip 1.5cm  \=\\ 
\> {\bf Tancredi Carli}  \> \> - \\
\> {\bf Richard Hawkings}  \> \> - \\
\> {\bf Michelangelo Mangano}  \>\> - \\
\> {\bf Patrick Janot}  \>\> - \\
\> {\bf Werner Riegler}  \> \> - \\
\> {\bf Daniel Froidevaux}  \> \> 2018   \\
\> {\bf Paolo Iengo}  \> \> 2016 \\
\> {\bf Aurelio Juste}  \> \> 2015 \\
\> {\bf Martine Bosman}  \> \> 2013 \\
\> {\bf Claude Guyot} \> \> 2009 

\end{tabbing}


%%%%%%%%%%%%%%%%%%%%%%%%%%%%%%%%%%%%%%%%%%%%%%%%%
%Talks at CONFERENCES
%%%%%%%%%%%%%%%%%%%%%%%%%%%%%%%%%%%%%%%%%%%%%%%%%
\vskip 0.8 cm
{\bf  \large  Talks at international conferences}
\vskip 0.3 cm
\begin{enumerate} 

\item "Higgs self-couplings at FCC-hh"\\
Higgs couplings 2018\\
Tokyo, Japan, November 2018

\item "Heavy Resonance searches at FCC-hh"\\
ICHEP 2018\\
Seoul, South Korea, July 2018

\item "Top physics at FCC"\\
ICHEP 2018\\
Seoul, South Korea, July 2018

\item "Design and performance studies of the calorimeter system for an FCC-hh experiment"\\
14th PISA meeting on advanced detectors\\
Isola d'Elba, Italy, May 2018

\item "Top quark production and top quark properties at the LHC"\\
Inaugural Conference Windows on the Universe Rencontres du Vietnam \\
Quy Nhon, Vietnam, August 2013.

\item "Top physics with the ATLAS detector"\\
Flavor Physics and CP Violation\\ 
Kibbutz Maale Hachamisha, Israel, May 2011

\end{enumerate}


%%%%%%%%%%%%%%%%%%%%%%%%%%%%%%%%%%%%%%%%%%%%%%%%%
%Talks at WORKSHOP
%%%%%%%%%%%%%%%%%%%%%%%%%%%%%%%%%%%%%%%%%%%%%%%%%
\vskip 0.8 cm
{\bf  \large  Talks at Workshops}
\vskip 0.3 cm
\begin{enumerate} 
\item "Searching for New Physics at future colliders: indirect versus direct reach"\\
CERN Faculty meeting\\
CERN, Switzerland, November 2018

\item "High mass di-muon interpretations"\\
Workshop on high-energy implications of flavor anomalies\\
CERN, Switzerland, October 2018

\item "Heavy resonances at 100TeV"\\
FCC week 2018\\
Amsterdam, Netherlands, April 2018

\item "FCC-hh analysis chain"\\
FCC week 2018\\
Amsterdam, Netherlands, April 2018

\item "Parametrized simulation and analysis"\\
FCC week 2016\\
Rome, Italy, April 2016

\item "Performance and potential of an ATLAS-like HCal Tile for FCC"\\
FCC week 2016\\
Rome, Italy, April 2016

\item "Performance requirements of the electromagnetic and hadronic calorimeters"\\
Workshop on requirements for future detector technologies in view of FCC-hh\\
CERN, Switzerland, February 2015

\item "Interpretation of Vector-like Quarks experimental results"\\
Workshop on Vector-like Quarks\\
DESY, Germany, September 2014

\item "Searches in single-lepton final states at ATLAS"\\
Workshop on Vector-like Quarks 2014 \\
DESY, Germany, September 2014

\item "Introduction to the FCC project"\\
GDR-Terascale workshop\\
 LLR Palaiseau, France, June 2014.

\item "First considerations about hadronic calorimetry"\\
$1^{st}$ Future Hadron Collider Workshop\\
CERN, Switzerland, June 2014.

\item "Progress report on DD4HEP/Geant4 based simulation"\\
$1^{st}$ Future Hadron Collider Workshop\\
CERN, Switzerland, June 2014.



\item "Using profile likelihood ratios at the LHC"\\
Top LHC-France workshop \\
Lyon, France, March 2013.

\item "ATLAS Exotic top and fourth generation searches"\\
LPCC, Implications of LHC results for TeV-scale physics\\
CERN, Switzerland, March 2012.

\item "Searches for fourth generation fermions at ATLAS"\\
Fourth fermion generation and single-top production Workshop\\
Leinsweiler Plafz, Germany, March 2012.

\item "$4^{th}$ generation searches in ATLAS and CMS"\\
GDR-Terascale workshop\\
CPPM, Marseille, France, October 2011


\item "Search for high mass resonances in the dimuon channel. Study of the impact of muon spectrometer alignment"\\
Physic ATLAS France workshop\\
 Biarritz, France, September 2007.	

\item "Alignment of the Muon spectrometer using the first cosmic straight tracks"\\
Muon Barrel Cosmic Run and Commissioning Workshop\\
CERN, Switzerland, December 2006.


\end{enumerate}


%%%%%%%%%%%%%%%%%%%%%%%%%%%%%%%%%%%%%%%%%%%%%%%%%%%%%%%%%%%%%%%%%%
%%%%%%%%%%%%%%%%%%%%%%%%%%%%%%%%%%%%%%%%%%%%%%%%%%%%%%%%%%%%%%%%%%


\vskip 0.8 cm
{\bf  \large  Seminars}

\vskip 0.3 cm

\begin{enumerate} 

\item "$4^{th}$ generation and heavy quark at LHC"\\
High Energy Accelerator Research Organization (KEK)\\
Tsukuba, Japan, April 2012.

\item "$4^{th}$ generation and heavy quark at LHC"\\
Laboratoire de Physique Subatomique et de Cosmologie (LPSC)\\
Grenoble, France, February 2012.

\item "$4^{th}$ generation and heavy quark at LHC"\\
Laboratoire d'Annecy le Vieux de Physique des Particules (LAPP)\\
Annecy, France, January 2012.

\item "$4^{th}$ generation and heavy quark at LHC"\\
Laboratoire de Physique Corpusculaire (LPC)\\
Clermont-Ferrand, France, January 2012.

\item "Search for high mass resonances in the dimuon channel with the ATLAS experiment"\\
Institut de F\'{\i}sica d'Altes Energies (IFAE)\\
Barcelona, Spain, September 2009

\end{enumerate}



\end{document}
